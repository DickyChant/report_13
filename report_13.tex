\documentclass{ctexart}
    \usepackage{mathrsfs}
    \usepackage{multirow}
    \usepackage{graphicx}
    \usepackage{array}
    \usepackage{makecell}
    \usepackage{amsmath}
    \usepackage{booktabs}
    \usepackage{float}
    \usepackage{diagbox}
    \newcommand\mgape[1]{\gape{$\vcenter{\hbox{#1}}$}}
    \newcommand\Ronum[1]{\uppercase\expandafter{\romannumeral #1\relax}}
    \newcommand\ronum[1]{\romannumeral #1\relax}
    \author{钱思天\ 1600011388 No.8}
    \title{实验十七\ RLC电路的谐振现象 \ 实验报告}
    \begin{document}
      \maketitle
      \section{实验数据与处理}
      根据实验的初始设定,以及各仪器的误差计算公式如:$$\mbox{电容箱允差:}\pm0.65\%(0.01\mu F\mbox{档})$$
          $$\mbox{标准电感允差:}\pm0.1\%$$
          $$e_R=0.1\Omega$$      
      
          得关于已知物理量,其值如下表:
        % Table generated by Excel2LaTeX from sheet 'Sheet1'
\begin{table}[H]
  \centering
  \caption{已知物理量表}
    \begin{tabular}{|c|c|c|c|}\hline
    已知物理量 & 电容$C$ & 电感$L$ & 电阻$R$ \\\hline
    值     & $0.05\mu F$ & $0.1H$ & $100\Omega$ \\\hline
    允差    & $3.25\times 10^{-4}\mu F$ & $1\times 10^{-4}H$ & $0.1\Omega $ \\\hline
    \end{tabular}%
  \label{tab:addlabel}%
\end{table}%
      \subsection{谐振频率的测量}
      \subsubsection{实验数据列表}
        
经利萨茹图形完成谐振频率的确定,并通过数字万用表完成各待测电压值的测量,并根据各仪器允差的确定方法:
$$\mbox{万用表交流电压档允差:}\pm(0.2\%+\mbox{十个字})$$
得关于测量物理量有下表:
% Table generated by Excel2LaTeX from sheet 'Sheet1'
\begin{table}[H]
  \centering
  \caption{本实验测量物理量表}
    \begin{tabular}{|c|c|c|c|c|}\hline
    测量物理量 & 谐振频率$f_0$ & 电路总电压$U$ & 电阻电压$U_R$ & 电容电压$U_C$ \\\hline
    值     & $2.2600kHz$ & $0.7001V$` & $0.5256V$ & $7.313V$ \\\hline
    允差    & $0.001kHz$ & $2.4\times 10^{-3}V$ & $2.1\times10^{-3}$ & $1.5\times10^{-2}V$ \\\hline
    \end{tabular}%
  \label{tab:addlabel}%
\end{table}%
\subsubsection{计算$Q$值}
根据公式$$Q_1=\frac{1}{\omega_0R'C}=\frac{1}{2\pi\f_0R'C}$$

      \subsection{实验数据处理}
      \subsubsection{验证线性关系}
     
      则有$I_2>I_1$。
      \section{收获与感想}
      在预习这个实验的时候,我情不自禁地想起来高一时所做的,验证牛顿第二定律的实验,也是用重物的重力做外力并计算加速度。

      在我看来,这两个实验有很多相似的地方,譬如都要使加速度远小于重力加速度等。

      其实从实验研究的对象,也能感受到高中与大学所学习内容的区别,从可当作质点运动的整体平动,到刚体的转动,我们所学习的物理也更加高深了。

      此外,在本次实验中,我也感受到了自己某些实验能力还有不足,例如对秒表的掌控等,希望在以后的实验课程中,能够提高自己的实验能力。

\end{document} 